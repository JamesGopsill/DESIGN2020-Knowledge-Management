{
  \titlefont
  \small
  \begin{wrapfigure}{r}{0.2\textwidth}
    \raggedleft
    \vspace{-0.2cm}
    \includegraphics[width=1.8cm]{figs/design-logo.png}
  \end{wrapfigure}
  \noindent \textbf{INTERNATIONAL DESIGN CONFERENCE - DESIGN 2020}\\
  \url{https://doi.org/10.1017/dsd.2020.94}

  \vspace{2cm}

  \Large\noindent\textbf{CO-WORD GRAPHS FOR DESIGN AND MANUFACTURE KNOWLEDGE MAPPING}

  \vspace{1cm}

  \normalsize \noindent J.\ Gopsill\textsuperscript{1}, M.\ Humphrey\textsuperscript{2}, D.\ Thompson\textsuperscript{2} and E.\ Garcia\textsuperscript{2} \\[0.2cm]
  \footnotesize \noindent \textsuperscript{1}University of Bath, United Kingdom, \textsuperscript{2} National Composites Centre, United Kingdom \\[0.1cm]
  \footnotesize \noindent J.A.Gopsill@bath.ac.uk\\

  \begin{mdframed}[backgroundcolor=gray!20] 
    \normalsize \noindent \textbf{Abstract} \\
    \normalfont Design \& Manufacture Knowledge Mapping is a critical activity in medium-to-large organisations supporting many organisational activities.
    However, techniques for effective mapping of knowledge often employ interviews, consultations and appraisals.
    Although invaluable in providing expert insight, the application of such methods is inherently intrusive and resource intensive.
    This paper presents word co-occurrence graphs as a means to automatically generate knowledge maps from technical documents and validates against expert generated knowledge maps.
  \end{mdframed}

  \small \noindent \textit{Keywords: knowledge management, technology development, design informatics, graph theory}

  \vspace{0cm}
}
